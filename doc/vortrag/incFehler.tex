\section{Fehlermeldungen}

\begin{frame}
\frametitle{Fehlermeldungen -- Ausgangspunkt}
Die bisherige Fehlerbehandlung in SL war unzureichend
\begin{itemize}
\item Parser parst nur bis zum ersten Syntaxfehler, gibt aber bis
  dahin gültige Teile des Programms einfach weiter
\item Statt Fehlermeldungen werden Scala-Objekte ausgegeben
\item Teilweise unbehandelte Exceptions; der Compiler beendet sich
  mit Stacktrace
\end{itemize}
$\Rightarrow$ Absolut unzureichend für jedes Projekt, das mehr als 
ein paar Zeilen Code umfasst.
\end{frame}

\begin{frame}[containsverbatim=true]
\frametitle{Fehlermeldungen -- Format}
\begin{lstlisting}
/path/to/file.sl:5:1-23: Use of undefined type(s) in `Foo': `Foo.Bar'
\end{lstlisting}
Das allgemeine Format ist folgendes:
\begin{center}
\emph{Dateiname} : \emph{Zeile(n)} [: \emph{Spalte(n)}] : \emph{Fehlermeldung}
\end{center}
Lokalisierung mit Zeilen- und Spaltennummer nur mit Combinator-Parser
\end{frame}

\begin{frame}
\frametitle{Fehlerarten}
\begin{itemize}
\item Syntaktische Fehler
  \begin{itemize}
  \item Häufige Fehler haben eigene Produktionen im Parser
  \item Teilweise nur kryptische Fehlermeldungen der Parsec-Bibliothek
  \item Abbruch nach erstem syntaktischen Fehler
  \end{itemize}
\item Semantische Fehler
  \begin{itemize}
  \item Typfehler werden erkannt, Ort teils unintuitiv
  \item Doppelte Deklarationen, fehlende Definitionen, falsche Aritäten, etc.
    werden mit Ort(en) zurückgegeben
  \end{itemize}
\item Importfehler
  \begin{itemize}
  \item Moduldateien nicht vorhanden: Ort des Import-Statements
    sowie Suchpfad für Datei werden ausgegeben
  \item Zyklische Importe, Qualifizierter Import der Prelude
  \end{itemize}
\item Laufzeitfehler werden vom JavaScript-Interpreter behandelt
\end{itemize}
\end{frame}
